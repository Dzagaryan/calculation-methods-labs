\documentclass{article}
\usepackage[T2A]{fontenc}
\usepackage{epigraph}
\usepackage[english, russian]{babel} % языковой пакет
\usepackage{amsmath,amsfonts,amssymb} %математика
\usepackage{mathtools}
\usepackage[oglav,spisok,boldsect,eqwhole,figwhole,hyperref,hyperprint,remarks,greekit]{../../style/fn2kursstyle}
\usepackage[utf8]{inputenc}
\usepackage[]{tkz-euclide}
\usepackage{algpseudocode}
\usepackage{pgfplots}
\usepackage{tikz-3dplot}
\usepackage[oglav,spisok,boldsect,eqwhole,figwhole,hyperref,hyperprint,remarks,greekit]{./style/fn2kursstyle}
\usepackage{multirow}
\usepackage{supertabular}
\usepackage{multicol}
\usepackage{tikz}
\usepackage{pgfplots}
\usepackage{float}
\usepackage{graphicx}
\pgfplotsset{compat=1.9}
\usepackage[svgnames]{pstricks}
\usepackage{pst-solides3d} 
\graphicspath{{../../style/}}
  



\newcommand{\cond}{\mathop{\mathrm{cond}}\nolimits}
\newcommand{\rank}{\mathop{\mathrm{rank}}\nolimits}
% Переопределение команды \vec, чтобы векторы печатались полужирным курсивом
\renewcommand{\vec}[1]{\text{\mathversion{bold}${#1}$}}%{\bi{#1}}
\newcommand\thh[1]{\text{\mathversion{bold}${#1}$}}
%Переопределение команды нумерации перечней: точки заменяются на скобки
\renewcommand{\labelenumi}{\theenumi)}
\newtheorem{theorem}{Теорема}
\newtheorem{define}{Определение}
\tdplotsetmaincoords{60}{115}
\pgfplotsset{compat=newest}

\title{Прямые методы решения систем
линейных алгебраических уравнений}
\author{Н.\,О.~Акиньшин}
\group{ФН2-51Б}
\date{2024}
\supervisor{А.\,С.~Джагарян}



\begin{document}
    \maketitle
    \newpage
    \tableofcontents
    \newpage

    \section{Контрольные вопросы}
    \begin{enumerate}
        \item Почему условие $\|C\| < 1$ гарантирует сходимость итерационных методов?
        \item Каким следует выбирать итерационный параметр $\tau$ в ме-
        тоде простой итерации для увеличения скорости сходимо-
        сти? Как выбрать начальное приближение $x_0$?
        \item На примере системы из двух уравнений с двумя неизвест-
        ными дайте геометрическую интерпретацию метода метода
        Якоби, метода Зейделя, метода релаксации.
        \item При каких условиях сходятся метод простой итерации,
        метод Якоби, метод Зейделя и метод релаксации? Какую
        матрицу называют положительно определенной?
        \item Выпишите матрицу $C$ для методов Зейделя и релаксации.
        \item Почему в общем случае для остановки итерационного
        процесса нельзя использовать критерий $\|x^k - x^{k+1}\| < \varepsilon$?
        \item Какие еще критерии окончания итерационного процесса
        Вы можете предложить?
    \end{enumerate}
    
\end{document}

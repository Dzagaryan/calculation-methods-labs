\documentclass{article}
\usepackage[T2A]{fontenc}
\usepackage{epigraph}
\usepackage[english, russian]{babel} 
\usepackage{amsmath,amsfonts,amssymb} 
\usepackage{mathtools}
\usepackage{pdflscape}
\usepackage{graphicx} 

\begin{document}

\begin{landscape} 
    \begin{table}[h!]
        \centering
        \caption{Результаты исследования итерационных методов при {$\varepsilon = 10^{-4}$}}
        \resizebox{\textwidth}{!}{ 
        \begin{tabular}{|c|c|p{2cm}|p{2cm}|p{2cm}|c|c|c|c|c|c|}
        \hline
        \textbf{Метод} & $\|C\|$ & \textbf{Оценка для числа итераций $k_\text{est}$} & \textbf{Норма ошибки после $k_\text{est}$ операций} & \textbf{Число итераций, необходимых для решения с точностью $\varepsilon$} & \multicolumn{2}{c|}{\textbf{Критерий 1}} & \multicolumn{2}{c|}{\textbf{Критерий 2}} & \multicolumn{2}{c|}{\textbf{Критерий 3}} \\
        \cline{6-11}
         & & & & & \textbf{Итерации} & \textbf{Ошибка} & \textbf{Итерации} & \textbf{Ошибка} & \textbf{Итерации} & \textbf{Ошибка} \\
        \hline
        Простые итерации ($\tau = ...$) & & & & & & & & & & \\
        \hline
        Якоби & & & & & & & & & & \\
        \hline
        Зейделя & & & & & & & & & & \\
        \hline
        Релаксации ($\omega = ...$) & & & & & & & & & & \\
        \hline
        Релаксации ($\omega = ...$) & & & & & & & & & & \\
        \hline
        \end{tabular}
        } 
    \end{table}
\end{landscape}

\end{document}